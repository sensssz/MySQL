%!TEX root = predictability.tex

\section{Main Sources of Latency Variance in MySQL}

\subsection{Experiment Setup}
We used the framework proposed in the previous section on MySQL to find out its
main sources of latency variance. We ran MySQL with the default configuration
on two Amazon EC2 m3.large instances running Ubuntu 14.04 LTS, each with 2
Intel Xeon E5-1670v2 2.5GHz virtual CPUs and 7.5GiB memory. One of them had a
20GB magnetic disk while the other ones ran on a 20GB SSD. We also had another
Amazon EC2 m3.medium instance to send queries to these two instances. This
instance had 1 Intel Xeon E5-1670v2 2.5GHz virtual CPU, 3.75GiB memory and a
10GB SSD. We used OLTPBenchmark to run the original TPC-C benchmark on this
instance and sent transactions to MySQL. We used MySQL version 5.6.21 in our
experiments. In the experiments, we set the value of k to 3 and the threshold
to be 5\% of the variance of latency.

\subsection{Main Sources of Variance}
The main sources of variance we found using the framework we proposed in
section 3 are shown in table \ref{tab:source-hdd} and \ref{tab:source-ssd}.
Note that the ones marked with an asterisk(*) are specific instances of the
corresponding factors. Although we set k to 3, we actually found only two
sources of variance for both. This is because the contribution of some of
the \textit{factors} with high responsibility values are too small, ie.
below the threshold(5\% of the overall variance), and therefore are not
selected.

\begin{table}[h!]
\centering
\begin{tabular}{|c|c|} \hline
Source&Contribution\\ \hline
\texttt{Var(buf\_pool\_mutex\_enter)}*&32.92\%\\ \hline
\texttt{Var(img\_btr\_cur\_search\_to\_nth\_level)}&8.3\%\\
\hline\end{tabular}
\caption{Sources of Variance on HDD}
\label{tab:source-hdd}
\end{table}

\begin{table}[h!]
\centering
\begin{tabular}{|c|c|} \hline
Source&Contribution(\%)\\ \hline
\texttt{Var(buf\_pool\_mutex\_enter)*}&16.89\%\\ \hline
\texttt{Var(lock\_wait\_suspend\_thread)*}&16.91\%\\
\hline\end{tabular}
\caption{Sources of Variance on SSD}
\label{tab:source-ssd}
\end{table}

\subsubsection{\texttt{Var(buf\_pool\_mutex\_enter)}}
Note that this main source of variance is an \textit{instance} of a
\textit{factor}, instead of the \textit{factor} itself. As its name
suggests, this function is used to acquire the lock on the buffer pool.
The buffer pool structure contains many related variables, and several
lists containing the buffer pages. There are many possible operations on
the buffer pool, so this function is used in a lot of different places for
various purposes. Because of its generality, looking at this function by
itself does not give us much useful information about the cause of its
variance. We look towards its \textit{instances} instead, and found one 
with a very high contribution. This function is called within the another 
function \texttt{buf\_page\_make\_young}, which is used to move a page
in the buffer pool to the head of the list of buffer pages. InnoDB uses a
variant of the Least Recently Used(LRU) algorithm as its buffer page
replacement algorithm. The buffer pages in the buffer pool are organized in a
list called \texttt{LRU} list in InnoDB. As we can easily tell from its name,
the pages in this list are kept in the LRU order - descending order of the
latest access time of each page. Therefore, when a buffer page is accessed, it
becomes the most recently used page, and should be moved to the head of the
\texttt{LRU} list(InnoDB uses a variant of the LRU algorithm, so this would not
happen every time a page is accessed). Since MySQL has a thread for each user,
a lock on this list must be acquired before any operation can be executed. This
is where the function \texttt{buf\_pool\_mutex\_enter} comes in. Therefore, the
variance here is actually the variance of the time function \texttt{buf\_page\_make\_young} spends on waiting for the other functions to release the lock.

\subsubsection{\texttt{Var(os\_event\_wait)}}
Like the previous one, this source of variance is also a specific instance
instead of a general factor. \texttt{os\_event} is a high level abstraction of
condition variables on different platforms, which are used in multi-threaded
programs for threads to wait for some particular conditions. Therefore, this
function can also be used for a lot of different purposes. This specific
instance is used in a function called \texttt{lock\_wait\_suspend\_thread},
which is related to the locking mechanism of InnoDB. InnoDB implements the
Multi-Granularity Locking(MGL) mechanism. Innodb have different types of locks,
including shared locks(S), exclusive locks(X), intention shared locks(IS) and
intention exclusive locks(IX). Shared locks are used when transactions need to
read a row while exclusive locks permit transactions to update or delete a row.
Some transactions, depending on what they do, may need to acquire a lock on an
entire table instead of a single record. To allow locks of different
granularity to exist simultaneously, InnoDB introduces intention locks into the
system. Before a transaction acquires a shared lock on a row in a table, it
must first acquire an intention shared lock or a stronger lock on that table.
Similarly, before it acquires an exclusive lock on a row in a table, it must
first acquire the intention exclusive lock on that table. The compatibility of
these different types of locks are shown in table \ref{tab:compatibility}('+'
means compatible, and '-' means conflict). Only when the type of lock a
transaction tries to acquire is compatible with the existing lock will the
required lock be granted. These locks also work for indexes.

However, locks are expensive and conflicts do not always occur when
transactions are executed. To reduce the overhead of locking, InnoDB introduces
implicit locks into the system. Each record has a field called \texttt{trx\_id}.
Before a transaction performs any operation on a record, InnoDB checks
if the transaction associated with the \texttt{trx\_id} is active. If not, the
\texttt{trx\_id} of the current transaction is written to the record, and the
intended operation can be carried out immediately without acquiring the lock.
If the transaction is still, the implicit lock is converted into an explicit
lock(which means acquiring an explicit lock of the appropriate type on the
record). If the intended lock type of the current transaction conflicts with
the explicit lock, the current transaction has to wait for the existing lock to
be released. Innodb uses the function \texttt{os\_event\_wait} to suspend the
thread.

\subsubsection{\texttt{Var(img\_btr\_cur\_search\_to\_nth\_level)}}
\texttt{img\_btr\_cur\_search\_to\_nth\_levelhis} is actually not a real
function, but rather the \textit{imaginary function} of a function called
\texttt{btr\_cur\_search\_to\_nth\_level}. An \textit{imaginary function} is,
as we mentioned before, a virtual function we create to represent the the work
of a function done by the execution of simple statements without invoking other
functions. The calculation of its time is fairly easy - subtract the sum of the
execution time of all \textit{child functions} from that of the \textit{parent
function}. The parent function, which is \texttt{btr\_cur\_search\_to\_nth\_level}, is a general function for searching in a given index tree and placing 
a tree cursor on a given level, leaving a shared lock or exclusive lock on the
cursor page. This is a rather high level function, and there's one important
loop in this function for searching the index tree until it reaches a certain
level. A level here corresponds to the height of a node in the tree, and leaf
nodes have level 0. Therefore, the reason behind the variance of this
\textit{imaginary function} could be attributed to the variance in the desired
level.

\begin{table}
\centering
\begin{tabular}{|c|c|c|c|c|} \hline
&S&X&IS&IX\\ \hline
S & + & - & + & -\\ \hline
X & - & - & - & -\\ \hline
IS & + & - & + & +\\ \hline
IX & - & - & + & +\\
\hline\end{tabular}
\caption{Lock Type Compatibility}
\label{tab:compatibility}
\end{table}


